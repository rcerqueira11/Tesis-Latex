\chapter{Marco Aplicativo}


\section{Descripción general de la solución} 
\setlength{\parskip}{5mm}
\setlength{\parskip}{0mm}


\section{Aplicación de la metodología Scrum con Agilus } 
\setlength{\parskip}{5mm}
\setlength{\parskip}{0mm}


\section{Requerimientos del Sistema } 
\setlength{\parskip}{5mm}
\setlength{\parskip}{0mm}


\section{Perfiles de Usuario} 
\setlength{\parskip}{5mm}

\textbf{-Administrador}: Es el usuario encargado de gestionar tanto usuarios como a los clientes y propietarios de los vehículos inspeccionados.
	\begin{itemize}
		\item Busqueda de usuarios del sistema.
		\item Busqueda de titulares de los vehiculos.
		\item Busqueda de personas trajeron el vehiculo a revision. 
		\item Edición de titulares de vehiculos.
		\item Edición de personas trajeron el vehiculo a revision.
		\item Edición y desactivación de usuarios Inspectores.
		\item Edición y desactivación de usuarios Taquilla.
		\item Reinicio de usuarios enviadoles nuevas contraseñas.
	\end{itemize}

\textbf{-Inspector}: Este usuario puede crear una cuenta en el sistema para gestionar solicitudes de inspección de vehículo. Tambien puede editar la información de su cuenta, ver la información de las solicitudes cerradas. 
	\begin{itemize}
		\item Creación y edición de una cuenta con su información básica para el uso del sistema.
		\item Visualizacion de las solicitudes de inspección pendientes por realizar.
		\item Vaciado de la informacion de la inspección del vehiculo en el sistema.
		\item Gestion de la solicitud de inspección (verificación y guardado). 
		\item Visualizar informacion de las solicitudes de inspección.
		\item Busqueda de solicitudes de inspección realizadas.
	\end{itemize}

\textbf{-Taquilla}: Este usuario puede crear una cuenta en el sistema para gestionar los tickets de atención, de las persona que requieran realizar una inspección.
	\begin{itemize}
		\item Creación y edición de una cuenta con su información básica para el uso del sistema.
		\item Creacion de ticket para solicitud de inspección.
		\item Cancelar un ticket abierto de una solicitud de inspección 
		\item Busqueda de tickets abiertos
	\end{itemize}


\setlength{\parskip}{0mm}


\section{Descripción del flujo asociado a la solución} 
\setlength{\parskip}{5mm}
\setlength{\parskip}{0mm}

\subsection{Proceso de creación de solicitud}
\setlength{\parskip}{5mm}
\setlength{\parskip}{0mm}

\subsection{Proceso de llenado de planilla de inspección}
\setlength{\parskip}{5mm}
\setlength{\parskip}{0mm}

\subsection{Proceso de gestión de solicitud de inspección}
\setlength{\parskip}{5mm}
\setlength{\parskip}{0mm}


\section{Análisis del modelo de datos y definición} 
\setlength{\parskip}{5mm}
\setlength{\parskip}{0mm}

\subsection{Listado de tablas de la aplicación}

\setlength{\parskip}{5mm}

A continuación se presenta un listado con todas las tablas de la base de datos perteneciente al sistema desarrollado, junto con una breve descripción para cada una:


\textbf{-registro\_status}: Permite clasificar el estado del usuario (Activo,Inactivo).

\textbf{-registro\_tipousuario}: Permite clasificar a los usuarios por tipo (Administrador, Cliente, Taquilla).

\textbf{-registro\_usuario}: Permite almacenar los usuarios en el sistema.

\textbf{-rcs\_accesoriosbase}: Permite almacenar los accesorios básicos de la planilla.

\textbf{-rcs\_accesoriosvehiculo}: Tabla utilizada para representar los accesorios base de un vehículo.

\textbf{-rcs\_condicionesgeneralesbase}: Permite almacenar las condiciones generales básicas de la planilla.

\textbf{-rcs\_condicionesgeneralesvehiculo}: Tabla utilizada para representar las condiciones generales base de un vehículo.

\textbf{-rcs\_detallesdatos}: Permite almacenar lo detalles y datos de la planilla.

\textbf{-rcs\_documentospresentados}: Permite almacenar los documentos presentados básicos de la planilla.

\textbf{-rcs\_documentospresentadosbase}: Tabla utilizada para representar los documentos presentados base de un vehículo.

\textbf{-rcs\_estadosolicitud}: Permite almacenar los diferentes estado que puede poseer la planilla (Abierta, Por Revisar,Cerrada).

\textbf{-rcs\_estadovehiculo}: Permite almacenar los diferentes estado que puede poseer las partes del vehículo (Bueno, Regular,Malo).

\textbf{-rcs\_mecanicabase}: Permite almacenar las mecánicas básicas de la planilla.

\textbf{-rcs\_mecanicavehiculo}: Tabla utilizada para representar las mecánicas base de un vehículo.

\textbf{-rcs\_motivosolicitud}: Permite almacenar los diferentes motivos por lo cual se realiza la solicitud de inspección.

\textbf{-rcs\_solicitudinspeccion}: Permite almacenar las solicitudes de inspección realizadas por el sistema.

\textbf{-rcs\_tipomanejo}: Permite almacenar los diferentes tipos de manejo (Automático, Sincrónico).

\textbf{-rcs\_tipovehiculo}: Permite guardar la información de los diferentes tipos de vehículos (Sedan, Carga, Coupe)

\textbf{-rcs\_titularvehiculo}: Permite almacenar la información referente a los titulares de los vehículos.

\textbf{-rcs\_trajovehiculo}: Permite almacenar la información de las personas que trajeron el vehículo a inspección en lugar del titular del titular.

\textbf{-rcs\_vehiculo}: Permite almacenar la información referente al vehículo.

\textbf{-rcs\_vehiculo\_accesorios\_vehiculo}: Tabla utilizada para representar las relaciones ManyToMany de rcs\_accesoriosvehiculo y vehículo.

\textbf{-rcs\_vehiculo\_condiciones\_generales\_vehiculo}: Tabla utilizada para representar las relaciones ManyToMany de rcs\_condicionesgeneralesvehiculo y vehículo.

\textbf{-rcs\_vehiculo\_detalles\_datos}: Tabla utilizada para representar las relaciones ManyToMany de rcs\_detallesdatos y vehículo.

\textbf{-rcs\_vehiculo\_documentos\_presentados}: Tabla utilizada para representar las relaciones ManyToMany de rcs\_documentospresentados y vehículo.

\textbf{-rcs\_vehiculo\_mecanica\_vehiculo}: Tabla utilizada para representar las relaciones ManyToMany de rcs\_mecanicavehiculo y vehículo.

\textbf{-django\_migrations}: Permite almacenar y llevar un control de las migraciones aplicadas para cambiar la base de datos a lo largo del proceso de desarrollo.

\textbf{-django\_session}: Permite almacenar todos los valores usados en la sesión de los usuarios dentro de la aplicación.


\setlength{\parskip}{0mm}





\subsection{Modelo de datos}
\setlength{\parskip}{5mm}
\setlength{\parskip}{0mm}


\section{Servicios web maybe} 
\setlength{\parskip}{5mm}
\setlength{\parskip}{0mm}

\section{Descripción de los módulos del sistema y sus interfaces } 
\setlength{\parskip}{5mm}
\setlength{\parskip}{0mm}


\section{Fase de pruebas maybe } 
\setlength{\parskip}{5mm}
\setlength{\parskip}{0mm}

\subsection{Pruebas funcionales}
\setlength{\parskip}{5mm}
\setlength{\parskip}{0mm}

\subsection{Pruebas de aceptación}
\setlength{\parskip}{5mm}
\setlength{\parskip}{0mm}
