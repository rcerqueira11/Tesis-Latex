\chapter{Marco Aplicativo}


\section{Descripción general de la solución} 
\setlength{\parskip}{5mm}
\setlength{\parskip}{0mm}


\section{Aplicación de la metodología Scrum}

\subsection{Lista de Objetivos (Pila del Producto)}
\setlength{\parskip}{5mm}


\begin{table}[htb]	
\begin{center}
\begin{tabular}{ | m{8cm} | m{8cm}| m{8cm}| } 
 \hline
 Sprint & Actividad & Fecha \\
 \hline
 1. & 
 \begin{itemize}
 	\item Instalación de las herramientas de desarrollo necesarias.
 	\item Configuración de los ambientes de desarrollo.
 	\item Maquetado de las Interfaces
 	\item Elaboración del Diagrama de Modelo relacional para la base de datos.
 \end{itemize}


 & faltafecha.\\
 \hline
 2. & 
 \begin{itemize}
 	\item Desarrollo del modulo de registro.
 	\item Realización de pruebas de funcionalidad.
 \end{itemize}


 & faltafecha.\\
 \hline
 3. & 
 \begin{itemize}
 	\item Desarrollo de la funcionalidad de recuperar contraseña
 	\item Desarrollo del modulo de gestión de solicitudes
 	\item Desarrollo de la funcionalidad de solicitar inspección
 	\item Realización de pruebas de funcionalidad
 \end{itemize}


 & faltafecha.\\
 \hline
 4. & 
 \begin{itemize}
 	\item Desarrollo de la funcionalidad de gestión de tickets de solicitudes de inspección
 	\item Desarrollo de la funcionalidad de consultar solicitudes de inspección
 	\item Realización de pruebas de funcionalidad
 \end{itemize}


 & faltafecha.\\
 \hline
 5. & 
 \begin{itemize}
 	\item Desarrollo de la funcionalidad de realizar inspección
 	\item Desarrollo de la funcionalidad de gestionar inspección
 	\item Desarrollo de la funcionalidad de generación de pdf
 	\item Realización de pruebas de funcionalidad
 \end{itemize}


 & faltafecha.\\
 \hline
 6. & 
 \begin{itemize}
 	\item Desarrollo del modulo administrativo
 	\item Desarrollo de la funcionalidad de Gestión de Usuarios (Taquilla Inspectores)
 	\item Realización de pruebas de funcionalidad
 \end{itemize}


 & faltafecha.\\
 \hline
 7. & 
 \begin{itemize}
 	\item Desarrollo de la funcionalidad de Gestión de Titulares de Vehículo
 	\item Desarrollo de la funcionalidad de Gestión de Personas que trajeron el Vehículo a la inspección
 	\item Realización de pruebas de funcionalidad
 \end{itemize}


 & faltafecha.\\
 \hline
 8. & 
 \begin{itemize}
 	\item Desarrollo del Vehículo en formato SVG
 	\item Realización de pruebas de funcionalidad
 \end{itemize}


 & faltafecha.\\
 \hline
\end{tabular}
\caption{Pila_del_Producto}
\label{Tabla:15}
\end{center}
\end{table}	



\setlength{\parskip}{0mm}


\subsection{Lista de tareas de la iteración (Pila del Sprint)}
\setlength{\parskip}{5mm}

\textbf{-Sprint 1}: 


\textbf{-Sprint 2}: En este Sprint se desarrollaron parte de las funcionalidades pertinentes al modulo de registro, con sus validaciones respectivas para el correcto funcionamiento del sistema de registro. Luego se realizacion las pruebas de funcionalidad correspondientes para validar el funcionamiento esperado del sistema.

\textbf{-Sprint 3}: En este Sprint se terminaron de desarrollar las funcionalidades restantes del modulo de registro, se empezo el desarrollo del modulo de gestion de solicitudes, en el cual se implementaron parte de las funcionalidades tales como el solicitar una inspección. Ademas se realizaron las pruebas pertinentes para el correcto funcionamiento de la aplicacion.

\textbf{-Sprint 4}: Se realizaron las funcionalidades de gestion de tickets de solicitud de inspección y consultar solicitudes de inspección, en esta ultima se obtuvieron las especificaciones de los campos por los cuales se filtrarian dichas solicitudes. Luego se realizaron las pruebas necesarias para validar el correcto funcionamiento de las funcionalidades desarrolladas.

\textbf{-Sprint 5}: En este Sprint se desarrollaron las funcionalidades de realizar inspeccion y la gestion de las mismas, ademas de la generacion de la planilla en la cual se registran todos los datos de las especificaciones del vehículo e informacion relevante de la inspeccion. Se realizaron las pruebas funcionales para validar el correcto funcionamiento.

\textbf{-Sprint 6}: En este Sprint se desarrollo parte del modulo de admnistración, en especifico la funcionalidad de gestion de usuarios que comprende en editar, inactivar y eliminar usuarios. Ademas se realizaron las pruebas pertinentes para garatizar la correcta funcionalidad.

\textbf{-Sprint 7}: Se desarrollaron las funcionalidades faltantes del modulo de administracion, gestion de los titulares del vehículo y de las personas que trajeron el vehículo a inspeccionar. Seguidamente se realizaron las pruebas funcionales para validar un correcto funcionamiento.

\textbf{-Sprint 8}: Se desarrollo la imagen del carro en formato SVG y se validaron las partes pertinente para que cambiaran de color dependiendo del estado del vehículo. Se realizaron las pruebas correspondientes para finalizar la planilla de inspeccion.

\setlength{\parskip}{0mm}


\section{Requerimientos del Sistema } 
\setlength{\parskip}{5mm}
	De acuerdo a los requerimientos definidos por el cliente (Product Owner), se establecio las siguiente clasificacion:

	\textbf{-Requerimientos Funcionales}: 

	\begin{itemize}
		\item Registro de usuarios por roles.
		\item Recuperar contraseña.
		\item Edicion del perfil de usuario.
		\item Administracion de los usuarios del sistema.
		\item Administracion de los clientes del sistema (Titular, Persona que trajo el Vehículo).
		\item Creacion de tickets para la inspeccion del vehículo.
		\item Consultar tickets creados.
		\item Consultar solicitudes de inspeccion.
		\item Registro de los datos de la inspeccion del vehículo.
		\item Gestionar solicitudes de inspeccion.
		\item Creacion de una imagen de carro donde se muestren diferentes colores, para las partes del carro, dependiendo su estado.
	\end{itemize}

	\textbf{-Requerimientos no Funcionales}: 

	\begin{itemize}
		\item Validar las entradas de los datos para el correcto funcionamiento del sistema.
	\end{itemize}

\setlength{\parskip}{0mm}


\section{Perfiles de Usuario} 
\setlength{\parskip}{5mm}

\textbf{-Administrador}: Es el usuario encargado de gestionar tanto usuarios como a los clientes y propietarios de los vehículos inspeccionados.
	\begin{itemize}
		\item Busqueda de usuarios del sistema.
		\item Busqueda de titulares de los vehiculos.
		\item Busqueda de personas trajeron el vehículo a revision. 
		\item Edición de titulares de vehiculos.
		\item Edición de personas trajeron el vehículo a revision.
		\item Edición y desactivación de usuarios Inspectores.
		\item Edición y desactivación de usuarios Taquilla.
		\item Reinicio de usuarios enviadoles nuevas contraseñas.
	\end{itemize}

\textbf{-Inspector}: Este usuario puede crear una cuenta en el sistema para gestionar solicitudes de inspección de vehículo. Tambien puede editar la información de su cuenta, ver la información de las solicitudes cerradas. 
	\begin{itemize}
		\item Creación y edición de una cuenta con su información básica para el uso del sistema.
		\item Visualizacion de las solicitudes de inspección pendientes por realizar.
		\item Vaciado de la informacion de la inspección del vehículo en el sistema.
		\item Gestion de la solicitud de inspección (verificación y guardado). 
		\item Visualizar informacion de las solicitudes de inspección.
		\item Busqueda de solicitudes de inspección realizadas.
	\end{itemize}

\textbf{-Taquilla}: Este usuario puede crear una cuenta en el sistema para gestionar los tickets de atención, de las persona que requieran realizar una inspección.
	\begin{itemize}
		\item Creación y edición de una cuenta con su información básica para el uso del sistema.
		\item Creacion de ticket para solicitud de inspección.
		\item Cancelar un ticket abierto de una solicitud de inspección 
		\item Busqueda de tickets abiertos
	\end{itemize}


\setlength{\parskip}{0mm}


\section{Descripción del flujo asociado a la solución} 
\setlength{\parskip}{5mm}
\setlength{\parskip}{0mm}

\subsection{Proceso de creación de solicitud}
\setlength{\parskip}{5mm}
\setlength{\parskip}{0mm}

\subsection{Proceso de llenado de planilla de inspección}
\setlength{\parskip}{5mm}
\setlength{\parskip}{0mm}

\subsection{Proceso de gestión de solicitud de inspección}
\setlength{\parskip}{5mm}
\setlength{\parskip}{0mm}


\section{Análisis del modelo de datos y definición} 
\setlength{\parskip}{5mm}
\setlength{\parskip}{0mm}

\subsection{Listado de tablas de la aplicación}

\setlength{\parskip}{5mm}

A continuación se presenta un listado con todas las tablas de la base de datos perteneciente al sistema desarrollado, junto con una breve descripción para cada una:


\textbf{-registro\_status}: Permite clasificar el estado del usuario (Activo,Inactivo).

\textbf{-registro\_tipousuario}: Permite clasificar a los usuarios por tipo (Administrador, Cliente, Taquilla).

\textbf{-registro\_usuario}: Permite almacenar los usuarios en el sistema.

\textbf{-rcs\_accesoriosbase}: Permite almacenar los accesorios básicos de la planilla.

\textbf{-rcs\_accesoriosvehiculo}: Tabla utilizada para representar los accesorios base de un vehículo.

\textbf{-rcs\_condicionesgeneralesbase}: Permite almacenar las condiciones generales básicas de la planilla.

\textbf{-rcs\_condicionesgeneralesvehiculo}: Tabla utilizada para representar las condiciones generales base de un vehículo.

\textbf{-rcs\_detallesdatos}: Permite almacenar lo detalles y datos de la planilla.

\textbf{-rcs\_documentospresentados}: Permite almacenar los documentos presentados básicos de la planilla.

\textbf{-rcs\_documentospresentadosbase}: Tabla utilizada para representar los documentos presentados base de un vehículo.

\textbf{-rcs\_estadosolicitud}: Permite almacenar los diferentes estado que puede poseer la planilla (Abierta, Por Revisar,Cerrada).

\textbf{-rcs\_estadovehiculo}: Permite almacenar los diferentes estado que puede poseer las partes del vehículo (Bueno, Regular,Malo).

\textbf{-rcs\_mecanicabase}: Permite almacenar las mecánicas básicas de la planilla.

\textbf{-rcs\_mecanicavehiculo}: Tabla utilizada para representar las mecánicas base de un vehículo.

\textbf{-rcs\_motivosolicitud}: Permite almacenar los diferentes motivos por lo cual se realiza la solicitud de inspección.

\textbf{-rcs\_solicitudinspeccion}: Permite almacenar las solicitudes de inspección realizadas por el sistema.

\textbf{-rcs\_tipomanejo}: Permite almacenar los diferentes tipos de manejo (Automático, Sincrónico).

\textbf{-rcs\_tipovehiculo}: Permite guardar la información de los diferentes tipos de vehículos (Sedan, Carga, Coupe)

\textbf{-rcs\_titularvehiculo}: Permite almacenar la información referente a los titulares de los vehículos.

\textbf{-rcs\_trajovehiculo}: Permite almacenar la información de las personas que trajeron el vehículo a inspección en lugar del titular del titular.

\textbf{-rcs\_vehiculo}: Permite almacenar la información referente al vehículo.

\textbf{-rcs\_vehiculo\_accesorios\_vehiculo}: Tabla utilizada para representar las relaciones ManyToMany de rcs\_accesoriosvehiculo y vehículo.

\textbf{-rcs\_vehiculo\_condiciones\_generales\_vehiculo}: Tabla utilizada para representar las relaciones ManyToMany de rcs\_condicionesgeneralesvehiculo y vehículo.

\textbf{-rcs\_vehiculo\_detalles\_datos}: Tabla utilizada para representar las relaciones ManyToMany de rcs\_detallesdatos y vehículo.

\textbf{-rcs\_vehiculo\_documentos\_presentados}: Tabla utilizada para representar las relaciones ManyToMany de rcs\_documentospresentados y vehículo.

\textbf{-rcs\_vehiculo\_mecanica\_vehiculo}: Tabla utilizada para representar las relaciones ManyToMany de rcs\_mecanicavehiculo y vehículo.

\textbf{-django\_migrations}: Permite almacenar y llevar un control de las migraciones aplicadas para cambiar la base de datos a lo largo del proceso de desarrollo.

\textbf{-django\_session}: Permite almacenar todos los valores usados en la sesión de los usuarios dentro de la aplicación.


\setlength{\parskip}{0mm}





\subsection{Modelo de datos}
\setlength{\parskip}{5mm}
\setlength{\parskip}{0mm}


\section{Servicios web maybe} 
\setlength{\parskip}{5mm}
\setlength{\parskip}{0mm}

\section{Descripción de los módulos del sistema y sus interfaces } 
\setlength{\parskip}{5mm}
\setlength{\parskip}{0mm}


\section{Fase de pruebas maybe } 
\setlength{\parskip}{5mm}
\setlength{\parskip}{0mm}

\subsection{Pruebas funcionales}
\setlength{\parskip}{5mm}
\setlength{\parskip}{0mm}

\subsection{Pruebas de aceptación}
\setlength{\parskip}{5mm}
\setlength{\parskip}{0mm}
