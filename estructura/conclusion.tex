\chapter*{CONCLUSIÓN Y RECOMENDACIONES }

\addcontentsline{toc}{chapter}{CONCLUSIÓN Y RECOMENDACIONES}


\setlength{\parskip}{5mm}

Se ha podido desarrollar de manera satisfactoria un sistema de información que cumple con los requerimientos planteados para su realización.

En cuanto al análisis de las especificaciones del cliente, se logró un correcto levantamiento de información logrando así, definir un alcance especifico de cada uno de los procesos.

A través del alcance definido se pudo generar el modelos de datos, diseño de las interfaces y módulos los cuales fueron certificados por el usuario.

Logrando así desarrollar un sistema capaz de gestionar las planillas de inspección de una compañía de seguros. En este sistema los usuarios podrán gestionar las solicitudes de inspección en conjunto con los datos de los vehículos inspeccionados. Manteniendo así una persistencia digital de la información para un mejor rendimiento y desempeño en el manejo de ella, cumpliendo así con los objetivos previamente definidos.

En cuanto a las pruebas se logró la completa validación de las funcionalidades creadas, realizando pruebas de funcionalidad y de usabilidad cuyos resultados sirvieron de retroalimentación positiva, en pro del desarrollo de un producto de calidad 100\% funcional. Teniendo un producto de software comprobado y con sus funcionalidades 

La metodología de desarrollo ágil SCRUM utilizada para el desarrollo de este proyecto, contribuyo de manera significativa en le realización del mismo. Ya que posee una muy buena organización de roles los cuales desempeñan un papel dentro del proyecto para lograr una sinergia entre los actores de SCRUM. Por lo cual fue un gran aporte al desarrollo de la aplicación.

Entre las recomendaciones se sugiere incorporar módulo móvil para la captura de los datos de la inspección, de modo que esta información pueda ser enlazada con el sistema web a través de un web service. También incorporar un módulo de Inteligencia de Negocio de manera que se puedan analizar los datos existentes y crear estrategias enfocadas a la administración y creación de conocimiento.

Y por ultimo incorporar un módulo de gestión documental de manera que se pueda administrar el flujo de documentos de todo tipo en la organización, permitir la recuperación de información desde ellos, determinar el tiempo que los documentos deben guardarse, eliminar los que ya no sirven y asegurar la conservación indefinida de los documentos más valiosos, aplicando principios de racionalización y economía.


\setlength{\parskip}{0mm}