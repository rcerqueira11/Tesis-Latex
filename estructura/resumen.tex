
\begin{center}
	Universidad Central de Venezuela\\
	Facultad de Ciencias\\
	Escuela de Computación\\
	
\end{center}

\begin{center}
	\textbf{ DESARROLLO DE UNA SOLUCIÓN PARA EL PROCESO DE INSPECCIÓN DE VEHÍCULOS EN UNA COMPAÑÍA DE SEGUROS. }
\end{center}

\begin{flushright}
\textbf{Autor:}Ricardo Antonio Cerqueira Salazar, C.I V-19940594
\end{flushright}

\begin{flushright}
\textbf{Tutor:}Prof. Franky Uzcátegui Polo
\end{flushright}


\begin{center}
	\large{\textbf{RESUMEN}}
\end{center}

\justify

El presente trabajo especial de grado expone el desarrollo de un sistema de información con tecnologías web, el cual fue elaborado como el módulo web, del proyecto compuesto por una aplicación móvil y una aplicación web, cuya finalidad es de automatizar el proceso de inscripción de vehículos en una compañía de seguros. Para el desarrollo de este proceso es indispensable que el vehículo sea examinado por un perito el cual determinará las condiciones con las cuales fue entregado el vehículo y si el cliente puede o no optar por el servicio. Este proceso se conoce como peritaje es imperativo que el cliente se presente con el vehículo, solicite un ticket, espere a ser atendido por el perito que le fue asignado, el perito realice la inspección. Sin embargo, este proceso presenta una serie de inconveniente en cuanto a la persistencia de los datos y la administración se refiere, almacenamiento de la planilla en físico, gestión del proceso de aprobación no está automatizado, por lo que mediante este estudio se creó un sistema que permite una mejor gestión de las inspecciones realizadas.

En el desarrollo de este sistema de información, se aplicó las metodologías de Agilus y Scrum en conjunto.Tomando como base el desarrollo por interfaces de Agilus y como gestión de tiempo los sprints de Scrum.

Se utilizaron diversas herramientas para la creación de este sistema, entre ellas el framework Django para la creación de la aplicación web ya que permite el acceso a la información referente al proceso de inspección de vehículos, desde cualquier lugar donde se tenga acceso a la web, información la cual se encuentra almacenada en una base de datos PostgresSQL la cual se comunica la base de datos central de la compañía de seguros.

\begin{justify}
	\large{\textbf{PALABRAS CLAVE: }}Inspección de vehículos, Compañía de seguros, Centro de inspección, Gestión solicitudes inspección.
\end{justify}