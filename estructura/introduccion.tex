\chapter*{INTRODUCCIÓN}

\addcontentsline{toc}{chapter}{INTRODUCCIÓN}

\setlength{\parskip}{4mm}

En la actualidad el mercado asegurador venezolano se encuentra en una alta demanda. Debido a esto las empresas aseguradoras poseen un alto índice de clientes cuyo manejo de información es indispensable para dar un buen servicio.

El alto volumen de información manejada en la suscripción de vehículos de cada uno de los clientes genera un fuerte impacto administrativo, tanto de personal como de tiempo. Los peritos encargados de la inspección del los vehículos, diariamente manejan un gran volumen de planillas las cuales pueden llegar a traspapelarse.

Por lo cual se propone una solución computacional mediante la cual se plantea desarrollar una aplicación web que permitirá el almacenamiento de dichas planillas de suscripción, para llevar un mejor manejo de las mismas. Administrarlas en caso de una re-inspección y poder gestionar mejor tramites como: atención del cliente, suscripción al seguro e inspección del vehículo. Garantizando así un mejor desempeño y velocidad en la atención de los clientes generando así una mayor satisfacción para los mismos.

La estructura de este documento consiste de las siguientes secciones: El primer capitulo el Marco Teórico consiste de los conceptos y definiciones sobre los temas y tecnologías utilizadas para el desarrollo de este estudio. 

El segundo capitulo Marco Metodológico consta de las diferente metodologías tanto ágiles como tradicionales para el desarrollo de software. En este capitulo se plantearan las diferentes características entre ambas tecnologías mostrando sus ventajas y desventajas. Además se expondrá un cuadro comparativo en el cual se podrá establecer cual metodología es la mas conveniente a utilizar.

El tercer capitulo al ser una solución que abarca el peritaje de vehículos, se incluyen conceptos y términos asociados con el área de seguro, peritaje e inspecciones de vehículos.

Por ultimo en el cuarto capitulo se presenta la propuesta de Trabajo Especial de Grado, mostrando los objetivos que se desean logra al culminar, en conjunto con la propuesta de solución, su justificación y el alcance de la misma. Además se especificaran las diferentes tecnologías que se utilizaran para el desarrollo de la propuesta.
\setlength{\parskip}{0mm}